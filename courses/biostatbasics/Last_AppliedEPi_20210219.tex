\documentclass{beamer} 
\usetheme{Madrid}
\usepackage{pdfpages}
\usepackage[utf8x]{inputenc}
\usepackage{url}
\usepackage{graphicx}
\usepackage{graphics}
\usepackage{adjustbox}
\usepackage{ragged2e}
\usepackage{soul}
\usepackage{MnSymbol,wasysym}
\usepackage{hyperref}
\usepackage{verbatim}
\usepackage{textcomp}

\usepackage{lmodern,textcomp}
 \setbeamertemplate{enumerate items}[default]
 \setbeamertemplate{itemize items}[circle]
 \setbeamertemplate{frametitle continuation}{}
\setbeamertemplate{section in toc}[circle]
\setbeamertemplate{subsection in toc}[circle]

 
%%%%%%%%%%%%%%%Bibliography setting%%%%%%%%%%%%%%%
\usepackage[round,comma,numbers]{natbib}
\setcitestyle{comma, numbers,sort&compress, super}\bibliographystyle{/Users/Enoch/survbib/bibtex/bst/vancouv12}
\setbeamertemplate{bibliography item}{text}

%%%%%%%%%%%%%%\begin{document}%%%%%%%%%%%%%%%%

%%%%%%%%%%%%%%%Title page%%%%%%%%%%%%%%%

\title{Last words to students}
\subtitle{Applied Epidemiology I}
\date{February 19, 2021}
\author[Enoch Chen]{Enoch Chen}
\institute[MEB, Karolinska Insitutet]{Department of Medical Epidemiology and Biostatistics \\Karolinska Insitutet}

%%%%%%%%%%%%%%\begin{document}%%%%%%%%%%%%%%%%

\begin{document}

\begin{frame}
\maketitle 
\end{frame}

%%%%
\section{Suggestions to your presentations}
\begin{frame}{\secname}
\begin{itemize}
	\item Format
	\begin{enumerate}
		\item  Italic font is usually for publication titles.
		\item  Less is more. Simplicity could also give high quality. 
		\item Slow is new fast. Going slowlier (with sufficient info) could make you audience understand faster.
		\item Re-design your tables/figures in the presentation. It is okay to have big tables in the written report, since people have time to read them. In contrast, not in slides.
		\item Don't put too much info in one slide. (I couldn't read.)
		\item Page(slide) number is important, since the audience can refer to the number at QA session.
		\item Separate different sessions. E.g., don't put methods and results into one slide.
		\item Running title could help remembering your research topic.
	\end{enumerate}
\end{itemize}
\end{frame}
%%%
\begin{frame}{\secname}
\begin{itemize}

	\item Introduction 
	\begin{enumerate}
 		\item Should sufficiently lead your audience
 		\item Touch upon the contents in the research questions
 		\item The last paragraph/part usually indicates the knowledge gap.
 \end{enumerate}
 	\item Aim and Research question
 	\begin{enumerate}
 		\item Only present aim $\rightarrow$ Fine to include research questions
 		\item Present both aim and research questions $\rightarrow$ do not mix them, but separate them.	
 	\end{enumerate}
	
	
	\item Results
	\begin{enumerate}
		\item Tables: as less grid (horizontal) lines as possible. Self-explanatory title!
		\item Titles and captions are essential components of tables/figures.
		\item Clearly present the results in a good manner. See Tables and Interpreting results: Principles. 
		\item Calculation on proportion should show not only the percentage but also how you generated it.
	\end{enumerate}
	\end{itemize}
\end{frame}

\begin{frame}{\secname}
\begin{itemize}
	\item Discussion
	\begin{enumerate}
		\item Refer to the results, give some quantitative values. e.g., Poorer self-rated health among young adults (mean: 2.5) was observed in this cohort.
		\item On discussion of bias, you should not only list those bias (selection bias, recall bias) but also say to which direction it might influence your results. Give informative bias.
		\item Including recent topics is nice, e.g., COVID-19. But don't go off topic. 
	\end{enumerate}
	\item Conclusion
	\begin{enumerate}
		\item One to two points/sentences are sufficient.	
	\end{enumerate}
	
	\item Reference
	\begin{enumerate}
		\item Put citations in the presentation 	slides (not only the written report).
		\item Put citation at the end of your sentence to support your statement.
	\end{enumerate}

	\item Peer review is valuable.
\end{itemize}
\end{frame}

%%%%
\section{Suggestions to asking coding questions}
\begin{frame}{\secname}
\begin{enumerate}
	\item Read the manual first! Read the manual first! (-help- or google first)
	\item Attach reproducible codes and example data.
	\item Clear comments in your code. (You ask to take a look. But where? My time is valuable, so is yours.)
	\item Screen shot is not preferable. 
\end{enumerate}
\end{frame}

%%%%
\section{Encouragement}
\begin{frame}{\secname}
\begin{itemize}
	\item I myself was quite lost after my first year in Epi programme.
	\item "Life is not straight forward. No one forced me to do what I'm doing now."
\end{itemize}
\end{frame}

\begin{frame}{\secname}
\begin{itemize}
	\item "Humans are longing to put things right." by N. T. Wright (University of Oxford).
	\item Epidemiologists are longing to put things (relevant to health) right as well. 
	\item So that's why we compare, compare and compare. 
	\item Longing to know the truth/fact/justice.
\end{itemize}
\end{frame}

\begin{frame}{\secname}
\begin{itemize}
	\item Please fill in the course evaluation. All (harsh) comments and suggestions are welcome.
	\item Don't be shy to say hi or email to me (enoch.yitung.chen@ki.se). 
	\item I have broad interest in biostatistics with main focuses on survival analysis and health economics research in cancer.
	\item Might see you somewhere at KI or if you do your thesis at MEB.
\end{itemize}
\end{frame}

\end{document}