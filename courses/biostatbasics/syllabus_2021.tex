\documentclass{article}
\usepackage[utf8]{inputenc}
\usepackage{enumitem, xcolor, soul}
\usepackage{hyperref}

\sethlcolor{lightgray}
\setlist{nosep}
\usepackage[a4paper, margin=2.54cm]{geometry}

\newcounter{lecture}[section]
\newenvironment{lecture}[1][]{\refstepcounter{lecture}\par\medskip
   \noindent \large \textbf{Lecture~\thelecture #1: }}{\medskip}

\pagestyle{empty}

\begin{document}



% HEADER
\begin{flushleft}
    \rule[1mm]{\linewidth}{1.5pt} \\
    \large \textbf{Applied Epidemiology I: Spring 2021} \hfill \large Enoch Yi-Tung Chen \\
    \large  \textbf{Lab session using Stata}
    \hfill (enoch.yitung.chen@ki.se) \\
     \large Course website: \url{https://enochytchen.com/courses/biostatbasics/} \\
    \rule[1mm]{\linewidth}{1.5pt} \\
\end{flushleft}
\vspace{-8pt}

% LECTURE 1
\begin{lecture}
Data management \hfill \normalsize \textbf{}
\end{lecture}

\begin{itemize}
    \item Aims 
    \item Good folder structure, documents, Readme.txt, habits on coding
    \item Other do’s and don’ts
\end{itemize}

%% Lab work - partners

\vspace{5pt}


% LECTURE 2
\begin{lecture}
Data clearance \hfill \normalsize \textbf{}
\end{lecture}

\begin{itemize}
\item Set up working directory
\item Import and save data
        \begin{itemize}
            \item File types: .xls, .txt, .csv, .dta, .xpt
            \item \verb|save, replace|
        \end{itemize}
    \item Manage datasets: \\ \verb|merge, append|
    \item Get to know the data: \\ \verb|summarize, describe, codebook, list|
    \item Manage variables: \\ - Variable types: \verb|numeric, string, keep, drop| \\
        - Action: \verb|label, rename, recode, generate, replace| \\
        - Condition: \verb|sort, by, if, in| and other operators
\end{itemize}

\vspace{5pt}

% LECTURE 3 -- see if students need more summary statistics, or want time to work on their own or what they need help on
\begin{lecture}
Graphs \hfill \normalsize \textbf{}
\end{lecture}

\begin{itemize}
\item Graphs
    \begin{itemize}
        \item Bad examples, learning from errors
        \item Basics of making graphs
        \item Study map
        \item Histogram, bar chart, scatter plot, box plot, line graph
        \item Customisation: stratification, combine two graphs, export
    \end{itemize}
\end{itemize}

\vspace{5pt}

% LECTURE 4 -- exercises 
\begin{lecture}
Summary statistics, tables and interpreting results
\end{lecture}

\begin{itemize}
  \item Summary statistics
    \begin{itemize}
        \item Measures of central tendency: mean, median, mode
        \item Measures of dispersion: range, IQR, variance, standard deviation
    \end{itemize}
    \item Tables
        \begin{itemize}
        \item Bad example
        \item Basics of making tables
        \item One-way tables, two-by-two tables
        \item Stata tool for Epidemiology
    \end{itemize}
    \item Basic Epidemiology terms
            \begin{itemize}
        \item Rate vs. proportion
        \item Risk, risk difference, risk ratio
        \item Odds, odds ratio
    \end{itemize}
    \item Interpreting results
      \begin{itemize}
        \item Principles
        \item Ratio $>$ or $<$ 1, more examples
    \end{itemize}
    \item Calculate ratios using Stata
\begin{itemize}
        \item Risk ratio, odds ratio, incidence rate ratio
    \end{itemize}
\end{itemize}

\vspace{5pt}

% LECTURE 5
\begin{lecture}
Q \& A session \hfill \normalsize \textbf{}
\end{lecture}

\begin{itemize}
    \item Any last statistical questions or clarifications on the halfway of the assignment
\end{itemize}

\vspace{16pt}

% LECTURE 6
\begin{lecture}
Q \& A session \hfill \normalsize \textbf{}
\end{lecture}

\begin{itemize}
    \item Any last statistical questions or clarifications before the assignment
\end{itemize}

\end{document}
